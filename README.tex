% document type
\documentclass{article}

% format
\usepackage[letterpaper, margin = 1.5cm]{geometry}

% trees
\usepackage{ebproof}

% placement
\usepackage{float}

% header
\title {
    Práctica 2 \\
    El lenguaje \texttt{EAB} \\
    Semántica
}

\author {
    Sandra del Mar Soto Corderi \\
    Edgar Quiroz Castañeda
}

\date {
    11 de septiembre del 2019
}

\begin{document}
    \section{Semántica Dinámica}
    Define la semántica dinámica de los operadores booleanos y las relaciones.

    Para los operadores booleanos

    \begin{figure}[H]
        \centering

        \begin{prooftree}
            \infer 0 {not(bool[p]) \rightarrow bool[!p]}
        \end{prooftree}
        \qquad
        \begin{prooftree}
            \hypo{e \rightarrow e'}
            \infer 1 {not(e) \rightarrow not(e')}
        \end{prooftree}
        \\ \bigskip

        \begin{prooftree}
            \infer 0 {and(bool[v_1], bool[v_2]) \rightarrow bool[v_1\&\&b_2]}
        \end{prooftree}
        \qquad
        \begin{prooftree}
            \hypo{e_1 \rightarrow e_1'}
            \infer 1 {and(e_1, e_2) \rightarrow and(e_1', e_2)}
        \end{prooftree}
        \qquad
        \begin{prooftree}
            \hypo{e_2 \rightarrow e_2'}
            \infer 1 {and(v, e_2) \rightarrow and(v, e_2')}
        \end{prooftree}
        \\ \bigskip

        \begin{prooftree}
            \infer 0 {or(bool[v_1], bool[v_2]) \rightarrow bool[v_1||b_2]}
        \end{prooftree}
        \qquad
        \begin{prooftree}
            \hypo{e_1 \rightarrow e_1'}
            \infer 1 {or(e_1, e_2) \rightarrow or(e_1', e_2)}
        \end{prooftree}
        \qquad
        \begin{prooftree}
            \hypo{e_2 \rightarrow e_2'}
            \infer 1 {or(v, e_2) \rightarrow or(v, e_2')}
        \end{prooftree}
        \caption{Semántica dinámica de los operadores booleanos}
        \label{fig:semd_bool}
    \end{figure}

    Para las relaciones

    \begin{figure}[H]
        \centering

        \begin{prooftree}
            \infer 0 {lt(n[v_1], n[v_2]) \rightarrow bool[v_1<b_2]}
        \end{prooftree}
        \qquad
        \begin{prooftree}
            \hypo{e_1 \rightarrow e_1'}
            \infer 1 {lt(e_1, e_2) \rightarrow lt(e_1', e_2)}
        \end{prooftree}
        \qquad
        \begin{prooftree}
            \hypo{e_2 \rightarrow e_2'}
            \infer 1 {lt(v, e_2) \rightarrow lt(v, e_2')}
        \end{prooftree}
        \\ \bigskip

        \begin{prooftree}
            \infer 0 {gt(n[v_1], n[v_2]) \rightarrow bool[v_1>b_2]}
        \end{prooftree}
        \qquad
        \begin{prooftree}
            \hypo{e_1 \rightarrow e_1'}
            \infer 1 {gt(e_1, e_2) \rightarrow gt(e_1', e_2)}
        \end{prooftree}
        \qquad
        \begin{prooftree}
            \hypo{e_2 \rightarrow e_2'}
            \infer 1 {gt(v, e_2) \rightarrow gt(v, e_2')}
        \end{prooftree}
        \\ \bigskip

        \begin{prooftree}
            \infer 0 {eq(n[v_1], n[v_2]) \rightarrow bool[v_1==b_2]}
        \end{prooftree}
        \qquad
        \begin{prooftree}
            \hypo{e_1 \rightarrow e_1'}
            \infer 1 {eq(e_1, e_2) \rightarrow eq(e_1', e_2)}
        \end{prooftree}
        \qquad
        \begin{prooftree}
            \hypo{e_2 \rightarrow e_2'}
            \infer 1 {eq(v, e_2) \rightarrow eq(v, e_2')}
        \end{prooftree}
        \caption{Semántica dinámica de los operadores relacionales}
        \label{fig:semd_rel}
    \end{figure}

    \section{Semántica Estática}
    Define la semántica estática de los operadores booleanos y las relaciones.

    Para las operaciones booleanas.

    \begin{figure}[H]
        \centering

        \begin{prooftree}
            \hypo{\Gamma \vdash t : Bool}
            \infer 1 {\Gamma \vdash not(t) : Bool}
        \end{prooftree}
        \qquad
        \begin{prooftree}
            \hypo{\Gamma \vdash t_1 : Bool}
            \hypo{\Gamma \vdash t_2 : Bool}
            \infer 2 {\Gamma \vdash and(t_1, t_2) : Bool}
        \end{prooftree}
        \qquad
        \begin{prooftree}
            \hypo{\Gamma \vdash t_1 : Bool}
            \hypo{\Gamma \vdash t_2 : Bool}
            \infer 2 {\Gamma \vdash or(t_1, t_2) : Bool}
        \end{prooftree}
        \caption{Semántica estática de los operadores booleanos}
        \label{fig:seme_bool}
    \end{figure}

    Para las operaciones de relación.

    \begin{figure}[H]
        \centering

        \begin{prooftree}
            \hypo{\Gamma \vdash t_1 : Nat}
            \hypo{\Gamma \vdash t_2 : Nat}
            \infer 2 {\Gamma \vdash lt(t_1, t_2) : Bool}
        \end{prooftree}
        \qquad
        \begin{prooftree}
            \hypo{\Gamma \vdash t_1 : Nat}
            \hypo{\Gamma \vdash t_2 : Nat}
            \infer 2 {\Gamma \vdash gt(t_1, t_2) : Bool}
        \end{prooftree}
        \qquad
        \begin{prooftree}
            \hypo{\Gamma \vdash t_1 : Nat}
            \hypo{\Gamma \vdash t_2 : Nat}
            \infer 2 {\Gamma \vdash eq(t_1, t_2) : Bool}
        \end{prooftree}
        \caption{Semántica estática de los operadores booleanos}
        \label{fig:seme_rel}
    \end{figure}


\end{document}